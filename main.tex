\documentclass[letterpaper]{article} % DO NOT CHANGE THIS
\usepackage[submission]{aaai2026}  % DO NOT CHANGE THIS
\usepackage{times}  % DO NOT CHANGE THIS
\usepackage{helvet}  % DO NOT CHANGE THIS
\usepackage{courier}  % DO NOT CHANGE THIS
\usepackage[hyphens]{url}  % DO NOT CHANGE THIS
\usepackage{graphicx} % DO NOT CHANGE THIS
\urlstyle{rm} % DO NOT CHANGE THIS
\def\UrlFont{\rm}  % DO NOT CHANGE THIS
\usepackage{natbib}  % DO NOT CHANGE THIS AND DO NOT ADD ANY OPTIONS TO IT
\usepackage{caption} % DO NOT CHANGE THIS AND DO NOT ADD ANY OPTIONS TO IT
\frenchspacing  % DO NOT CHANGE THIS
\setlength{\pdfpagewidth}{8.5in} % DO NOT CHANGE THIS
\setlength{\pdfpageheight}{11in} % DO NOT CHANGE THIS
%
% These are recommended to typeset algorithms but not required. See the subsubsection on algorithms. Remove them if you don't have algorithms in your paper.
\usepackage{algorithm}
\usepackage{algorithmic}
\usepackage{amsmath,amsfonts,amssymb}

%
% These are are recommended to typeset listings but not required. See the subsubsection on listing. Remove this block if you don't have listings in your paper.
\usepackage{newfloat}
\usepackage{listings}
\DeclareCaptionStyle{ruled}{labelfont=normalfont,labelsep=colon,strut=off} % DO NOT CHANGE THIS
\lstset{%
	basicstyle={\footnotesize\ttfamily},% footnotesize acceptable for monospace
	numbers=left,numberstyle=\footnotesize,xleftmargin=2em,% show line numbers, remove this entire line if you don't want the numbers.
	aboveskip=0pt,belowskip=0pt,%
	showstringspaces=false,tabsize=2,breaklines=true}
\floatstyle{ruled}
\newfloat{listing}{tb}{lst}{}
\floatname{listing}{Listing}
%
% Keep the \pdfinfo as shown here. There's no need
% for you to add the /Title and /Author tags.
\pdfinfo{
/TemplateVersion (2026.1)
}

% DISALLOWED PACKAGES
% \usepackage{authblk} -- This package is specifically forbidden
% \usepackage{balance} -- This package is specifically forbidden
% \usepackage{color (if used in text)
% \usepackage{CJK} -- This package is specifically forbidden
% \usepackage{float} -- This package is specifically forbidden
% \usepackage{flushend} -- This package is specifically forbidden
% \usepackage{fontenc} -- This package is specifically forbidden
% \usepackage{fullpage} -- This package is specifically forbidden
% \usepackage{geometry} -- This package is specifically forbidden
% \usepackage{grffile} -- This package is specifically forbidden
% \usepackage{hyperref} -- This package is specifically forbidden
% \usepackage{navigator} -- This package is specifically forbidden
% (or any other package that embeds links such as navigator or hyperref)
% \indentfirst} -- This package is specifically forbidden
% \layout} -- This package is specifically forbidden
% \multicol} -- This package is specifically forbidden
% \nameref} -- This package is specifically forbidden
% \usepackage{savetrees} -- This package is specifically forbidden
% \usepackage{setspace} -- This package is specifically forbidden
% \usepackage{stfloats} -- This package is specifically forbidden
% \usepackage{tabu} -- This package is specifically forbidden
% \usepackage{titlesec} -- This package is specifically forbidden
% \usepackage{tocbibind} -- This package is specifically forbidden
% \usepackage{ulem} -- This package is specifically forbidden
% \usepackage{wrapfig} -- This package is specifically forbidden
% DISALLOWED COMMANDS
% \nocopyright -- Your paper will not be published if you use this command
% \addtolength -- This command may not be used
% \balance -- This command may not be used
% \baselinestretch -- Your paper will not be published if you use this command
% \clearpage -- No page breaks of any kind may be used for the final version of your paper
% \columnsep -- This command may not be used
% \newpage -- No page breaks of any kind may be used for the final version of your paper
% \pagebreak -- No page breaks of any kind may be used for the final version of your paperr
% \pagestyle -- This command may not be used
% \tiny -- This is not an acceptable font size.
% \vspace{- -- No negative value may be used in proximity of a caption, figure, table, section, subsection, subsubsection, or reference
% \vskip{- -- No negative value may be used to alter spacing above or below a caption, figure, table, section, subsection, subsubsection, or reference

\setcounter{secnumdepth}{2} %May be changed to 1 or 2 if section numbers are desired.

% The file aaai2026.sty is the style file for AAAI Press
% proceedings, working notes, and technical reports.
%

% Title

% Your title must be in mixed case, not sentence case.
% That means all verbs (including short verbs like be, is, using,and go),
% nouns, adverbs, adjectives should be capitalized, including both words in hyphenated terms, while
% articles, conjunctions, and prepositions are lower case unless they
% directly follow a colon or long dash
\title{Quantum Circuit Synthesis with Deep Reinforcement Learning and Heuristic Search}
\author{
    %Authors
    % All authors must be in the same font size and format.
    Written by AAAI Press Staff\textsuperscript{\rm 1}\thanks{With help from the AAAI Publications Committee.}\\
    AAAI Style Contributions by Pater Patel Schneider,
    Sunil Issar,\\
    J. Scott Penberthy,
    George Ferguson,
    Hans Guesgen,
    Francisco Cruz\equalcontrib,
    Marc Pujol-Gonzalez\equalcontrib
}
\affiliations{
    %Afiliations
    \textsuperscript{\rm 1}Association for the Advancement of Artificial Intelligence\\
    % If you have multiple authors and multiple affiliations
    % use superscripts in text and roman font to identify them.
    % For example,

    % Sunil Issar\textsuperscript{\rm 2},
    % J. Scott Penberthy\textsuperscript{\rm 3},
    % George Ferguson\textsuperscript{\rm 4},
    % Hans Guesgen\textsuperscript{\rm 5}
    % Note that the comma should be placed after the superscript

    1101 Pennsylvania Ave, NW Suite 300\\
    Washington, DC 20004 USA\\
    % email address must be in roman text type, not monospace or sans serif
    proceedings-questions@aaai.org
%
% See more examples next
}

%Example, Single Author, ->> remove \iffalse,\fi and place them surrounding AAAI title to use it
\iffalse
\title{My Publication Title --- Single Author}
\author {
    Author Name
}
\affiliations{
    Affiliation\\
    Affiliation Line 2\\
    name@example.com
}
\fi

\iffalse
%Example, Multiple Authors, ->> remove \iffalse,\fi and place them surrounding AAAI title to use it
\title{My Publication Title --- Multiple Authors}
\author {
    % Authors
    First Author Name\textsuperscript{\rm 1},
    Second Author Name\textsuperscript{\rm 2},
    Third Author Name\textsuperscript{\rm 1}
}
\affiliations {
    % Affiliations
    \textsuperscript{\rm 1}Affiliation 1\\
    \textsuperscript{\rm 2}Affiliation 2\\
    firstAuthor@affiliation1.com, secondAuthor@affilation2.com, thirdAuthor@affiliation1.com
}
\fi


% REMOVE THIS: bibentry
% This is only needed to show inline citations in the guidelines document. You should not need it and can safely delete it.
\usepackage{bibentry}
% END REMOVE bibentry

\begin{document}

\maketitle

\begin{abstract}
Quantum circuit synthesis is the process of implementing
a quantum algorithm with a given gate set. In this
paper, we first show how this problem can be posed as a
pathfinding problem. Next, we use DeepCubeA to learn a
heuristic function for quantum circuit synthesis for one to
three qubit circuits with deep reinforcement learning and
solve problem instances with batch weighed A* search. We
compare our approach against a state-of-the-art
quantum circuit synthesis tool and show that are our
approach is competitive.
\end{abstract}



%% SECTION Introduction
\section{Introduction}

The realization of quantum algorithms on real-world quantum computers is accomplished through
quantum compiling \cite{maronese2022quantum}, also known as quantum circuit synthesis.
Given a quantum algorithm, quantum compiling finds a quantum circuit that implements that algorithm.
A quantum algorithm can be represented as a matrix of size $2^n \times 2^n$, with $n$ being the number of qubits.
A quantum circuit is a sequence of gates that perform operations on one or more qubits,
where a qubit is an irreducible unit of quantum information.
Given a quantum circuit, the matrix that represents the algorithm it implements is obtained by starting with the
identity matrix and changing its entries according to matrix multiplications determined by the gates used in the circuit.
As a result, quantum compiling can be posed as a pathfinding problem, where the start state
is the identity matrix, the goal is a given quantum algorithm, the transitions are quantum gates, and the transition costs
are determined by a combination of quantum gate execution time and the noise the gate introduces.


Finding quantum circuits that implement a given algorithm is a non-trivial task that can sometimes take state-of-the-art
methods several hours, even for single qubit operators \cite{paradis2024synthetiq}.
Given that we can pose this as a pathfinding problem, we build on the DeepCubeA algorithm \cite{agostinelli2019solving}
and hindsight experience replay (HER) \cite{andrychowicz2017hindsight} to use deep reinforcement learning to learn
a heuristic function that maps a given quantum circuit and quantum goal algorithm to an estimate of the cost of a
shortest path (i.e. ``cost-to-go'') from the given circuit to a circuit that implements the goal algorithm.




%% SECTION Background
\section{Background}


\subsection{Quantum Computing}


\paragraph{Qubits and Gates}


\paragraph{Clifford+T Gate Set}


\paragraph{Unitary Synthesis}




%% SECTION Related work
\section{Related Work}




%% SECTION Approach
\section{Approach}




%% SECTION Experiments
\section{Experiments}




%% SECTION Discussion and Future Work
\section{Discussion and Future Work}




%% SECTION Conclusion
\section{Conclusion}



\appendix

\section{Hurwitz Encoding}

\begin{algorithm}[tb]
    \caption{
        Procedure for Hurwitz encoding of a unitary matrix $A$.
        Here $\Theta(k)$ represents initialization of an all zero array of size $k$,
        the operator $X ~@_{i,j}~ Y$ represents the block multiplication
        of a $2 \times 2$ matrix $X$ along rows $i,j$ of matrix $Y$,
        and the `diagonal' function selects the diagonal elements of a matrix.
    }

    \label{alg:hurwitz}

    \textbf{Input}: unitary matrix $A \in U(n)$ \\
    \textbf{Output}: real vectors $\theta,\phi \in [0, 2\pi)^k$, $\lambda \in [0, 2\pi)^n$

    \begin{algorithmic}[1]
        \STATE $k \gets \lfloor n(n-1)~/~2 \rfloor$
        \STATE $\theta \gets \Theta(k)$
        \STATE $\phi \gets \Theta(k)$
        \STATE $x \gets 1$ 
        \FOR {$i \in [1\dots(n-1)]$}
            \FOR {$j \in [n \dots (i+1)]$}
                \STATE $a \gets A_{i,i}$
                \STATE $b \gets A_{j,i}$
                \IF {$b \neq 0$}
                    \STATE $r \gets \sqrt{|a|^2+|b|^2}$
                    \STATE $c \gets |a| ~/~ r$
                    \STATE $s \gets |b| ~/~ r$
                    \IF {$a \neq 0$}
                        \STATE $\theta_x \gets \arctan (s~/~c)$
                        \STATE $\phi_x \gets \arg (a) - \arg (b)$
                    \ELSE
                        \STATE $\theta_x \gets \pi/2$
                        \STATE $\phi_x \gets - \arg (b)$
                    \ENDIF
                    \STATE $G \gets \begin{bmatrix} c & e^{i\phi_x}s \\ -e^{-i\phi_x}s & c \end{bmatrix}$
                    \STATE $A \gets G ~@_{i,j}~ A$
                \ENDIF
                \STATE $x \gets x + 1$
            \ENDFOR
        \ENDFOR
        \STATE $d \gets \text{diagonal}(A)$
        \STATE $\lambda \gets \arg(d)$
    \end{algorithmic}
\end{algorithm}

The generalized Euler angle encoding, or `Hurwitz' encoding of a unitary matrix is the unique decomposition
of a matrix $A \in U(n)$ into $n^2$ angles ranging from $0$ to $2\pi$ given by algorithm \ref{alg:hurwitz}.
Note that the pseudocode contains comparisons to 0, but in practice the values are compared within a tolerance value to ensure numerical stability.
For the purpose of quantum circuit synthesis, the global phase can also be removed by first projecting the unitary
matrix into a special unitary matrix according to the function $\Pi : U(n) \rightarrow SU(n)$, defined as
$\Pi (A) = e^{-i (\arg(\text{det}(A)))/n} A$.
The last angle in $\lambda$ can then be disregarded, and the resulting representation will contain $n^2-1$ angles.



% Uncomment the following to link to your code, datasets, an extended version or similar.
% You must keep this block between (not within) the abstract and the main body of the paper.
% \begin{links}
%     \link{Code}{https://aaai.org/example/code}
%     \link{Datasets}{https://aaai.org/example/datasets}
%     \link{Extended version}{https://aaai.org/example/extended-version}
% \end{links}

\bibliography{refs.bib,forest.bib}



\end{document}
